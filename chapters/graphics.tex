\section{Graphiken}

\subsection{Graphiken einbinden}

\begin{frame}
  \frametitle{Graphiken einbinden}
  Blubb
\end{frame}

\subsection{Graphiken erstellen}
\begin{frame}
  \frametitle{Graphiken erstellen}
  Mehrere Möglichkeiten
  \begin{itemize}
  \item \hologo{METAPOST}\\
    basiert auf \hologo{METAFONT}
  \item PSTricks\\
    verwendet PostScript für die Graphikerstellung
  \item Ti\emph{k}Z\\
    komplett \TeX-basierend
  \end{itemize}
\end{frame}

\begin{frame}[fragile]
  \frametitle{\hologo{METAPOST}}
  \begin{columns}[T]
    \begin{column}{.25\textwidth}
      \begin{mplibcode}
beginfig(1);
    numeric u; u=50pt;
    pickup pencircle scaled 2pt;
    draw fullcircle scaled 1u;
    draw (-.3u,-.1u)..(0,-.3u)..(.3u,-.1u);
    pickup pencircle scaled 4pt;
    draw (0,0);
    pickup pencircle scaled 7pt;
    draw (-.2u,.2u); draw (.2u,.2u);
endfig;
      \end{mplibcode}
    \end{column}
    \begin{column}{.75\textwidth}
      \examplefile{examples/graphics/smiley.mp}
      \lstinputlisting[language=MetaPost]{examples/graphics/smiley.mp}
    \end{column}
  \end{columns}
\end{frame}

\begin{frame}
  \frametitle{\hologo{METAPOST} (2)}
  \begin{itemize}
  \item Vorträge von DANTE2013 für Einsteiger:
    \begin{itemize}
    \item
      \href{http://www.dante.de/events/dante2013/Programm/Vortraege/folien-entenmann.pdf}{Walter
      Entenmann – Graphik mit \hologo{METAPOST}}
    \item
      \href{http://www.dante.de/events/dante2013/Programm/Vortraege/folien-voipio.pdf}{Mari
      Voipio – Entry-Level MetaPost}
    \end{itemize}
    \item Übersetzung mit \texttt{mpost smiley.mp}\\
    $\rightarrow$ \texttt{smiley.1} (ähnlich EPS-Graphik)
  \item Einbinden mit \texttt{\textbackslash includegraphics}\\
    Kein \hologo{pdfLaTeX}!  Umweg über PostScript notwendig
  \item Direkte Verwendung mit \hologo{LuaTeX} (\texttt{\href{http://ctan.org/pkg/luamplib}{CTAN://pkg/luamplib}})
  \end{itemize}
\end{frame}

\begin{frame}
  \frametitle{PSTricks}
  \begin{columns}[T]
    \begin{column}{.25\textwidth}
      \includegraphics[width=2cm]{smiley-pstricks}
    \end{column}
    \begin{column}{.75\textwidth}
      \examplefile{examples/graphics/smiley-pstricks.tex}
      \lstinputlisting[language={[LaTeX]TeX}]{examples/graphics/smiley-pstricks.tex}
    \end{column}
  \end{columns}
\end{frame}

\begin{frame}
  \frametitle{PSTricks (2)}
  \begin{itemize}
  \item Die Bibel: Herbert Voß – Grafik mit PostScript für \TeX{} und
    \LaTeX{}
  \item Übersetzung (um PDF zu erhalten) mit\\
    \texttt{\$ latex smiley-pstricks.tex}\\
    \texttt{\$ dvips smiley-pstricks.dvi}\\
    \texttt{\$ ps2pdf smiley-pstricks.ps}
  \item Funktioniert \emph{nicht} mit \hologo{LuaTeX}, nur
    \hologo{LaTeX} und \hologo{XeTeX}
  \item Über 100~Pakete auf CTAN
  \item Galerie auf \href{http://pstricks.tug.org}{pstricks.tug.org}
  \end{itemize}
\end{frame}

\begin{frame}
  \frametitle{Ti\emph{k}Z}
  \begin{columns}[T]
    \begin{column}{.25\textwidth}
      Smiley
    \end{column}
    \begin{column}{.75\textwidth}
      \examplefile{examples/graphics/smiley-pstricks.tex}
    \end{column}
  \end{columns}
\end{frame}

\begin{frame}
  \frametitle{Ti\emph{k}Z (2)}
  \begin{itemize}
  \item Kein Buch, nur PDF-Dokumentation
  \item Funktioniert mit allen \TeX-Interpretern
  \item Funktionsumfang ähnlich PSTricks
  \item Galerie auf \href{http://www.texample.net}{texample.net}
  \end{itemize}
\end{frame}

%%% Local Variables: 
%%% mode: latex
%%% coding: utf-8
%%% TeX-engine: luatex
%%% TeX-PDF-mode: t
%%% TeX-master: "../pr-ieee-main"
%%% End: 
