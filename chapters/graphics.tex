\section{Grafiken}

\subsection{Grafiken einbinden}

\begin{frame}
  \frametitle{Grafiken einbinden}
  \begin{itemize}
  \item Viele Möglichkeiten, oft nicht trivial
  \item hier Übersicht, weiteres liefert Paketdokumentation
  \item Häufig verwendete Pakete:
    \begin{itemize}
    \item \texttt{graphicx}\\
      \emph{Der} Standard, eigentlich immer nötig
    \item \texttt{subfig}\\
      Möglichkeit Abbildungen zu gruppieren
    \item \texttt{wrapfig}\\
      Bilder im Fließtext
    \end{itemize}
  \end{itemize}
\end{frame}

\begin{frame}[fragile]
  \frametitle{\texttt{graphicx}}
  \framesubtitle{Grundlagen}
  \begin{itemize}
  \item Einbinden \texttt{\textbackslash usepackage\{graphicx\}}
  \item Zentraler Befehl \texttt{\textbackslash
      includegraphics[options]\{filename\}}
  \item Mögliche Werte für \texttt{options}:\\
    \texttt{width}, \texttt{height}, \texttt{angle}\dots
  \item Normales Einbinden:
\begin{lstlisting}[language={[LaTeX]TeX}]
\begin{figure}[ht]
  \centering
  \includegraphics[options]{filename}
  \caption{Caption text}
  \label{img:num}
\end{figure}
\end{lstlisting}
  \end{itemize}
\end{frame}

\begin{frame}
  \frametitle{\texttt{graphicx}}
  \framesubtitle{Gleitumgebungen}
  \begin{itemize}
  \item \texttt{figure} sog. Gleitumgebung
  \item wird von \TeX{} positioniert
  \item Positionierungswunsch
    \begin{itemize}
    \item \texttt{h}: here
    \item \texttt{t}: top
    \item \texttt{b}: bottom
    \item \texttt{p}: special page
    \end{itemize}
  \end{itemize}
\end{frame}

\begin{frame}
  \frametitle{\texttt{graphicx}}
  \framesubtitle{Weitere nützliche Befehle}
  \begin{itemize}
  \item \texttt{\textbackslash rotatebox\{angle\}\{text\}}\\
    Rotiert \texttt{text} um Winkel \texttt{angle}
  \item \texttt{\textbackslash
      scalebox\{h-scale\}\{v-scale\}\{text\}}\\
    Bietet die Möglichkeit, den \texttt{text} zu skalieren
  \item \texttt{\textbackslash reflectbox\{text\}}\\
    Spiegelt \texttt{text}
  \end{itemize}
\end{frame}

\begin{frame}
  \frametitle{\texttt{graphicx}}
  \framesubtitle{Beispiel}
  \examplefile{examples/graphics/graphic.tex}
  \fbox{\includegraphics[width=.9\textwidth]{scr-graphic}}
\end{frame}

\begin{frame}[fragile]
  \frametitle{\texttt{subfig}}
  \begin{itemize}
  \item Grundsätzlicher Code
\begin{lstlisting}[language={[LaTeX]TeX}]
\begin{figure}[ht]
  \centering
  \subfloat[Caption1]{<img inclusion code>}
  \subfloat[Captionn]{<img inclusion code>}
  \caption{Caption for whole}
  \label{label-for-whole}
\end{figure}
\end{lstlisting}
  \item Alternativ auch mit \texttt{\textbackslash parbox} oder
    \texttt{minipage}-Umgebung
  \item Einsatz selten sinnvoll\dots
  \end{itemize}
\end{frame}

\begin{frame}
  \frametitle{\texttt{subfig}}
  \framesubtitle{Beispiel}
  \examplefile{examples/graphics/subfigure.tex}
  \fbox{\includegraphics[width=.9\textwidth]{scr-subfig}}
\end{frame}

\begin{frame}[fragile]
  \frametitle{\texttt{wrapfig}}
  \begin{itemize}
  \item Grundsätzlicher Code
\begin{lstlisting}[language={[LaTeX]TeX}]
\begin{wrapfigure}[lines]{placement}[overhang]{width}
  <figure>
\end{wrapfigure}
\end{lstlisting}
  \item Seltsame Effekte beim Seitenumbruch
  \item Empfehlung: Verzichten!
  \end{itemize}
\end{frame}

\begin{frame}
  \frametitle{\texttt{wrapfig}}
  \framesubtitle{Beispiel}
  \examplefile{examples/graphics/wrapfigure.tex}
  \fbox{\includegraphics[width=.9\textwidth]{scr-wrapfigure}}
\end{frame}

\subsection{Grafiken erstellen}
\begin{frame}
  \frametitle{Grafiken erstellen}
  Mehrere Möglichkeiten
  \begin{itemize}
  \item \hologo{METAPOST}\\
    basiert auf \hologo{METAFONT}
  \item PSTricks\\
    verwendet PostScript für die Graphikerstellung
  \item Ti\emph{k}Z\\
    komplett \TeX-basierend
  \end{itemize}
\end{frame}

\begin{frame}[fragile]
  \frametitle{\hologo{METAPOST}}
  \begin{columns}[T]
    \begin{column}{.25\textwidth}
      \begin{mplibcode}
beginfig(1);
    numeric u; u=50pt;
    pickup pencircle scaled 2pt;
    draw fullcircle scaled 1u;
    draw (-.3u,-.1u)..(0,-.3u)..(.3u,-.1u);
    pickup pencircle scaled 4pt;
    draw (0,0);
    pickup pencircle scaled 7pt;
    draw (-.2u,.2u); draw (.2u,.2u);
endfig;
      \end{mplibcode}
    \end{column}
    \begin{column}{.75\textwidth}
      \examplefile{examples/graphics/smiley.mp}
      \lstinputlisting[language=MetaPost]{examples/graphics/smiley.mp}
    \end{column}
  \end{columns}
\end{frame}

\begin{frame}
  \frametitle{\hologo{METAPOST} (2)}
  \begin{itemize}
  \item Vorträge von DANTE2013 für Einsteiger:
    \begin{itemize}
    \item
      \href{http://www.dante.de/events/dante2013/Programm/Vortraege/folien-entenmann.pdf}{Walter
      Entenmann – Graphik mit \hologo{METAPOST}}
    \item
      \href{http://www.dante.de/events/dante2013/Programm/Vortraege/folien-voipio.pdf}{Mari
      Voipio – Entry-Level MetaPost}
    \end{itemize}
    \item Übersetzung mit \texttt{mpost smiley.mp}\\
    $\rightarrow$ \texttt{smiley.1} (ähnlich EPS-Graphik)
  \item Einbinden mit \texttt{\textbackslash includegraphics}\\
    Kein \hologo{pdfLaTeX}!  Umweg über PostScript notwendig
  \item Direkte Verwendung mit \hologo{LuaTeX} (\texttt{\href{http://ctan.org/pkg/luamplib}{CTAN://pkg/luamplib}})
  \end{itemize}
\end{frame}

\begin{frame}
  \frametitle{PSTricks}
  \begin{columns}[T]
    \begin{column}{.25\textwidth}
      \includegraphics[width=2cm]{smiley-pstricks}
    \end{column}
    \begin{column}{.75\textwidth}
      \examplefile{examples/graphics/smiley-pstricks.tex}
      \lstinputlisting[language={[LaTeX]TeX}]{examples/graphics/smiley-pstricks.tex}
    \end{column}
  \end{columns}
\end{frame}

\begin{frame}
  \frametitle{PSTricks (2)}
  \begin{itemize}
  \item Übersetzung (um PDF zu erhalten) mit\\
    \texttt{\$ latex smiley-pstricks.tex}\\
    \texttt{\$ dvips smiley-pstricks.dvi}\\
    \texttt{\$ ps2pdf smiley-pstricks.ps}
  \item Funktioniert \emph{nicht} mit \hologo{LuaTeX}, nur
    \hologo{LaTeX} und \hologo{XeTeX}
  \item Über 100~Pakete auf CTAN
  \item Galerie auf \href{http://pstricks.tug.org}{pstricks.tug.org}
  \end{itemize}
  \begingroup
  \printbibliography[heading=none,keyword=graphic]
  \endgroup
\end{frame}

\begin{frame}
  \frametitle{Ti\emph{k}Z}
  \begin{columns}[T]
    \begin{column}{.25\textwidth}
      \includegraphics[width=2cm]{smiley-tikz}
    \end{column}
    \begin{column}{.75\textwidth}
      \examplefile{examples/graphics/smiley-tikz.tex}
      \lstinputlisting[language={[LaTeX]TeX}]{examples/graphics/smiley-tikz.tex}
    \end{column}
  \end{columns}
\end{frame}

\begin{frame}
  \frametitle{Ti\emph{k}Z (2)}
  \begin{itemize}
  \item Kein Buch, nur PDF-Dokumentation (726 Seiten)
  \item Funktioniert mit allen \TeX-Interpretern
  \item Funktionsumfang ähnlich PSTricks
  \item Galerie auf \href{http://www.texample.net}{texample.net}
  \item Diversen Erweiterungspakete
  \item Viele Beispiele auch auf \href{http://tex.stackexchange.com}{\TeX.sx}
  \end{itemize}
\end{frame}

%%% Local Variables: 
%%% mode: latex
%%% coding: utf-8
%%% TeX-engine: luatex
%%% TeX-PDF-mode: t
%%% TeX-master: "../pr-ieee-main"
%%% End: 
