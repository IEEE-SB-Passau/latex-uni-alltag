\section{Pr\"asentationen}

\begin{frame}
  \frametitle{Präsentationen}
  \begin{itemize}
  \item Präsentation: Häufig notwendig
  \item Folien oft grausam (PowerPoint-Effekt-Wahn)
  \item Mehrere \LaTeX-Pakete
  \item Inhalt vor Design!
  \item (Fast) alle \LaTeX-Möglichkeiten für Präsentation
  \end{itemize}
  \begingroup
  \printbibliography[heading=none,keyword=presentation]
  \endgroup
\end{frame}

\subsection{powerdot}
\begin{frame}
  \frametitle{\texttt{powerdot}}
  \begin{itemize}
  \item Basiert auf PSTricks
  \item \texttt{latex} → \texttt{dvips} → \texttt{ps2pdf}
  \item Überschaubarer Funktionsumfang
  \item Freie Gestaltungsmöglichkeit
  \item \href{http://ctan.org/pkg/powerdot}{\texttt{CTAN://pkg/powerdot}}
  \end{itemize}
\end{frame}

\begin{frame}[fragile]
  \frametitle{\texttt{powerdot}}
  \framesubtitle{Code-Beispiel}
  \examplefile{examples/presentations/minimal-powerdot.tex}
  \lstinputlisting[language={[LaTeX]TeX}]{examples/presentations/minimal-powerdot.tex}
\end{frame}

\subsection{beamer}
\begin{frame}
  \frametitle{\texttt{beamer}}
  \begin{itemize}
  \item Basiert auf Ti\emph{k}Z
  \item Funktinioniert mit allen \TeX-Engines
  \item Riesiger Funktionsumfang (245~Seiten Dokumentation)
  \item Layout-Erstellung oft nicht trivial
  \item \href{http://ctan.org/pkg/beamer}{\texttt{CTAN://pkg/beamer}}
  \end{itemize}
\end{frame}

\begin{frame}
  \frametitle{\texttt{beamer}}
  \framesubtitle{Code-Beispiel}
  \examplefile{examples/presentations/minimal-beamer.tex}
  \lstinputlisting[language={[LaTeX]TeX}]{examples/presentations/minimal-beamer.tex}
\end{frame}

\begin{frame}
  \frametitle{\texttt{beamer}}
  \framesubtitle{Konzepte}
  \begin{itemize}
  \item Folienlayout stark modularisiert
    \begin{itemize}
    \item Theme, Inner Theme, Outer Theme, Color Theme, Font Theme
    \item beliebig kombinierbar
    \item Viele Standardthemes (siehe Doku)
    \end{itemize}
  \item Overlays – Überlagerung von mehreren Folien
    \begin{itemize}
    \item \texttt{\textbackslash pause}
    \item \texttt{\textbackslash onslide<Overlays>\{Text\}}
    \item \texttt{\textbackslash only<Overlays>\{Text\}}
    \item \texttt{\textbackslash visible<Overlays>\{Text\}},
      \texttt{\textbackslash invisible<Overlays>\{Text\}}
    \item \texttt{\textbackslash uncover<Overlays>\{Text\}}
    \item \texttt{\textbackslash alert<Overlays>\{Text\}}
    \item Sparsam verwenden, lenkt oft mehr ab
    \end{itemize}
  \end{itemize}
\end{frame}

\begin{frame}[fragile]
  \frametitle{\texttt{beamer}}
  \framesubtitle{Vordefinierte Umgebungen}
  \texttt{theorem, corollory, definition, definitions, fact, proof,
    example}\dots
  \begin{columns}[T]
    \begin{column}{.35\textwidth}
      \setbeamertemplate{blocks}[rounded][shadow=true]
      \setbeamercolor{block body}{bg=normal text.bg!90!black}
      \setbeamercolor{block title}{bg=normal text.bg!90!red}
      \setbeamercolor{block title}{parent={structure,block body}}
      \begin{theorem}[Pythagoras]
        \[ c^{2} = a^{2} + b^{2} \]
      \end{theorem}
    \end{column}
    \begin{column}{.6\textwidth}
\begin{lstlisting}[language={[LaTeX]TeX}]
\setbeamertemplate{blocks}[rounded][shadow=true]
\setbeamercolor{block body}{bg=normal text.bg!90!black}
\setbeamercolor{block title}{bg=normal text.bg!90!red}
\setbeamercolor{block title}{parent={structure,block body}}
\begin{theorem}[Pythagoras]
  \[ c^{2} = a^{2} + b^{2} \]
\end{theorem}
\end{lstlisting}
    \end{column}
  \end{columns}
\end{frame}

\begin{frame}[fragile]
  \frametitle{\texttt{beamer}}
  \framesubtitle{Vordefinierte Umgebungen (2)}
\begin{lstlisting}[language={[LaTeX]TeX}]
\begin{beamercolorbox}[Optionen]{Farbe} ...
    \end{beamercolorbox}
\begin{beamerboxesrounded}[Optionen]{Farbe} ...
    \end{beamerboxesrounded}
\begin{block}<Overlays>{Kopfzeile} ... \end{block}
\begin{block}{Kopfzeile}<Overlays> ... \end{block}
\begin{alertblock}<Overlays>{Kopfzeile} ... \end{alertblock}
\begin{alertblock}{Kopfzeile}<Overlays> ... \end{alertblock}
\begin{exampleblock}<Overlays>{Kopfzeile} ...
    \end{exampleblock}
\begin{exampleblock}{Kopfzeile}<Overlays> ...
    \end{exampleblock}
\end{lstlisting}
\end{frame}

\begin{frame}[fragile]
  \frametitle{\texttt{beamer}}
  \framesubtitle{Mehrspaltiger Satz}
\begin{lstlisting}[language={[LaTeX]TeX}]
\begin{columns}[Optionen]
  \begin{column}[Platzierung]{Breite} ...
      \end{column}
  \begin{column}[Platzierung]{Breite} ...
      \end{column}
  ...
\end{columns}
\end{lstlisting}
  Optionen für \texttt{columns}
  
  \begin{tabular}{ll}
    \textbf{Name} & \textbf{Bedeutung} \\\hline
    \texttt{b} & Unterste Basislinie der Spalten\\
    \texttt{c} & Vertikale Zentrierung (Standard)\\
    \texttt{t} & Basislinie der ersten Zeile\\
    \texttt{T} & Oberkante der ersten Zeile
  \end{tabular}
\end{frame}
%%% Local Variables: 
%%% mode: latex
%%% coding: utf-8
%%% TeX-engine: luatex
%%% TeX-PDF-mode: t
%%% TeX-master: "../pr-ieee-main"
%%% End: 
