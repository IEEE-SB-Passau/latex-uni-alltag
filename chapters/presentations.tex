\section{Pr\"asentationen}

\begin{frame}
  \frametitle{Präsentationen}
  \begin{itemize}
  \item
  \end{itemize}
\end{frame}

\subsection{powerdot}
\begin{frame}
  \frametitle{\texttt{powerdot}}
  \begin{itemize}
  \item Basiert auf PSTricks
  \item \texttt{latex} → \texttt{dvips} → \texttt{ps2pdf}
  \item Überschaubarer Funktionsumfang
  \item Freie Gestaltungsmöglichkeit
  \item \href{http://ctan.org/pkg/powerdot}{\texttt{CTAN://pkg/powerdot}}
  \end{itemize}
  \begingroup
  \printbibliography[heading=none,keyword=presentation]
  \endgroup
\end{frame}

\begin{frame}[fragile]
  \frametitle{\texttt{powerdot}}
  \framesubtitle{Code-Beispiel}
  \examplefile{examples/presentations/minimal-powerdot.tex}
  \lstinputlisting[language={[LaTeX]TeX}]{examples/presentations/minimal-powerdot.tex}
\end{frame}

\subsection{beamer}
\begin{frame}
  \frametitle{\texttt{beamer}}
  \begin{itemize}
  \item Basiert auf Ti\emph{k}Z
  \item Funktinioniert mit allen \TeX-Engines
  \item Riesiger Funktionsumfang (245~Seiten Dokumentation)
  \item Layout-Erstellung oft nicht trivial
  \item \href{http://ctan.org/pkg/beamer}{\texttt{CTAN://pkg/beamer}}
  \end{itemize}
  \begingroup
  \printbibliography[heading=none,keyword=presentation]
  \endgroup
\end{frame}

\begin{frame}
  \frametitle{\texttt{beamer}}
  \framesubtitle{Code-Beispiel}
  \examplefile{examples/presentations/minimal-beamer.tex}
  \lstinputlisting[language={[LaTeX]TeX}]{examples/presentations/minimal-beamer.tex}
\end{frame}
%%% Local Variables: 
%%% mode: latex
%%% coding: utf-8
%%% TeX-engine: luatex
%%% TeX-PDF-mode: t
%%% TeX-master: "../pr-ieee-main"
%%% End: 
